\section{Introduction}

Signals flowing through computational units to sculpt decision paths underlie all intelligent systems.
In brains, the computational units are nerve cells that signal each other by activating the synapses between neighboring cells via abrupt electrochemical changes known as \emph{action potentials}, commonly called \emph{spikes}.
% Spikes are the communication protocol of the brain \cite{rieke_spikes:_1999}. 
% Despite this seemingly simple cascading protocol where spikes activate synapses of target neurons, which, in turn, impacts those neurons' spiking, we have not unearthed the ``neural code''.
The spikes in one set of neurons activate synapses in another set of neurons, which, in turn, affects the spiking of their associated neurons, inducing ripples of spiking activity across the system.
While this cascading protocol is seemingly simple, we have not unearthed the ``neural code'' of how brains compute.
Indeed, brains likely use spikes in a variety of ways to regulate body functions, induce behaviors, and realize intelligence.

While we are inspired by the brain and explore some of its spiking mechanisms and properties, our intent is not to reverse-engineer the brain.
Rather, we investigate, more broadly and abstractly, various computations with spikes.
In particular, we focus on systems built with generic computational units that are sensitive to \emph{coincident} and \emph{precisely-timed} events.
%we focus oncharacterize the properties, constraints, and possibilities of spiking neural networks
%adopt the notion of computational units signaling each other via spikes and focus on systems built with computational units sensitive to \emph{coincident} and \emph{precisely-timed} spikes.
%, explore the generic properties, constraints, and possibilities of 
We present a theory for how artificial systems can and how brains may use neurons that are sensitive to \emph{coincident} spikes.
We also show how such neurons can self-organize into networks that learn and exhibit generally intelligent behaviors in real-time.

\subsection{Thesis outline}

We begin with various definitions and the components that make up any intelligent system.
We then cast the computational elements of an intelligent system as sensors whose inputs and outputs are information.
Critically, information operates as a pattern for the sensors.
In more complex systems, the outputs from earlier sensors serve as more meaningful patterns for subsequent sensors.
Such sequencing is the mechanism by which intelligent systems build semantic representations.

We then show how spiking neural networks (SNNs)---networks composed of neurons that communicate with their neighbors via spikes---are ideally suited for intelligent learning systems, especially those systems that operate in real-time, because these neurons are simultaneously pattern detectors and generators that utilize a common protocol to provide multiple semantic representations in dynamic settings.
We describe our rationale for not pursuing a rate-based neural code and focus on systems sensitive to the synchronous spikes of multiple neurons onto a target neuron as the source of information sensing and transmission.

To develop coincident-sensitive neurons and networks, we start with homogenous point neurons; homogenous in that the population of neurons have the same statistical firing patterns and ``point'' meaning that they have no morphology such as dendrites.
We construct measures of coincidence from simulations with these neurons.
We then construct rules governing the dynamics of a set of spiking neurons that will be used.
These rules describe a neuron's internal state over time, how it responds to incoming spikes, and how it propagates spikes to its targets.
Given that we want to promote the recognition of coincident spikes, we develop those learning rules and show how spiking networks can self-organize.

Given our findings from point neuron networks, we explore the impact of modulating a neuron's spiking sensitivity given its spiking history.
For example, if it has not fired an action potential recently, it can become more sensitive so that it does spike.
Contrastingly, if it has been quite active and has been rapidly emitting spikes, we can dial its sensitivity down.
Relatedly, we explore modulating the emitted spike efficacy onto a neuron's targets.
In this area, we evaluate the effect and possibilities of networks where even when a spike is generated, it is not communicated to all targets.
As well, we consider setting some targets to receive more spikes than others depending on the source neuron's spiking history.
In this case, some targets may receive most of the source neuron's spikes while other targets may only receive spikes after the source neuron has generated several rapid spikes.

We also explore possible roles of dendrites.
We describe how coincident-based detection and processing can be optimized with neurons with dendritic trees that accentuate positive signals with superadditive rules and particular branching structures.

In all of our networks, our neurons follow Dale's Law that loosely states that a neuron acts completely excitatory or inhibitory to all of its synaptic targets \cite{dalePharmacologyNerveEndings1935}.
(It cannot excite some targets and inhibit others. It has to be one or the other.)
This allows us to explore the roles of excitation and inhibition more distinctly.
More to point, we argue that inhibition is not overtly to decrease the firing rate in targets as much as it is to help coordinate synchronous spikes between them. 
Another option for inhibition is to gate propagating signals entirely.

We then show some working examples of these spiking networks, starting with simple toy problems, to image problems, NLP datasets, and in robotics.

And finally, we discuss the ramifications of this theory.
We claim that meaning is self-emergent in learning systems.
This means that a learning system's inputs, its internal computing capabilities, and the way it is taught are critical to how it derives meaning.
A spiking neuron need not be passive and a network of active, spiking neurons is able to generate and create novel behaviors.
We realize it in our own lives, living under a spiking brain.
Therefore, we claim this a Strong AI, closely resembling the way biological brains work.
As such, there are hypotheses for neuroscience that need further testing and opportunities to develop AIs that interface with humans and behave in a more human-like way.

\subsection{Intelligence and the elements of an intelligent system}

We subscribe to \href{https://en.wikipedia.org/wiki/Intelligence}{Wikipedia's definition of \emph{intelligence}} \cite{wikipediacontributorsIntelligenceWikipediaFree2021}, which, at the time of this writing, is:
\begin{displayquote}
...[T]he ability to perceive or infer information, and to retain it as knowledge to be applied towards adaptive behaviors within an environment or context.
\end{displayquote}

Given this definition, any intelligent system must contain the following:
\begin{enumerate}
  \item \textbf{Sensors.} ``The ability to perceive or infer information'' requires the intelligent system to have sensing capabilities to recognize or differentiate information.
We label any mechanism that performs this function a sensor.
This may be a programmatic operation or rule; when executed, it ``senses'' an input and provides an output.
A binary logic gate, for example, operates as a sensor since it monitors for a particular input pattern and emits TRUE or FALSE accordingly.
We can also say that simple sensors, such as logic gates or rules, can aggregate into more complex groupings, such as circuits, to form more specialized and nuanced sensing.
Such meta-sensors will be able to perceive or infer more complex forms of information.

The role of sensing is to translate information into more meaningful and operational representations, possibly by transducing signals through other sensors in an encoding and decoding paradigm. 
  \item \textbf{Memory.} ``...to retain [information] as knowledge'' requires the intelligent system to have some means of storing information, making this content available to inform subsequent actions.
  \item \textbf{Rational output/action.} ``...to be applied towards adaptive behaviors within an environment or context'' means that an intelligent system provides an output that makes reasonable sense given the context.
This could be a simple decision or some sort of complex behavior depending on the system's capabilities.

It may also be that the system's outputs are not overtly exposed to an external observer.
Such behaviors may be strategizing, planning, imagining counterfactuals, or predicting possible futures.
Since these behaviors are \emph{informational}---they add internal meta-data and knowledge---the system may have a recursive or hierarchical characteristic; there must be sensors and memory to deal with the internal meta-information as well as the external information.

% The particular way information is represented and stored will also be a constraint on how the system can behave.
\end{enumerate}

\subsection{Information, patterns, and meaning}

Within our definition of \emph{intelligence}, let us be clearer about \emph{information}.
The Oxford English dictionary defines \emph{information} as ``What is conveyed or represented by a particular arrangement or sequence of things \cite{oxforduniversitypressInformation2021}.''
We will adopt this definition of information and restate it more simply as ``a pattern and its meanings.''
Note that this definition of information, on its face, is not the same as Shannon information, frequently used in digital designs, which equates information as the rarity of a message/pattern and does not directly consider \emph{meaning}. 
Specifically, with respect to a communication system, Shannon stated (emphasis theirs):
\begin{displayquote}
Frequently the messages have \emph{meaning}; that is they refer to or are correlated according to some system with certain physical or conceptual entities. These semantic aspects of communication are irrelevant to the engineering problem. The significant aspect is that the actual message is one \emph{selected from a set} of possible messages \cite{shannonMathematicalTheoryCommunication1948}.
\end{displayquote}
% As alluded above, we interpret Shannon's term \emph{message} as a type of pattern.
Indeed, communication systems such as those Shannon was focused on, chartered with the engineering task of optimally \emph{transmitting} messages, need not consider meaning.
His focus was on the digitization of a pattern and the computations to encode, decode, transform, and transmit its bits; not to construct an intelligent system.

While embedding or encoding \emph{meaning} may not be required in many computational tasks, meaning is critical to intelligent systems.
Using the Oxford English Dictionary definition of \emph{information} above, \emph{meaning} is the ``conveyed or represented'' part.
This means that intelligent systems go beyond pattern detection to also encode context and representations.
There is a comprehension of where a pattern fits in relation to other patterns.
But where a given thing fits relative to other things is simply ``a particular arrangement or sequence of things'' that we also recognize as a pattern.
In short, \textbf{meaning is a pattern; it is a \emph{pattern of patterns.}}
This is the most important point of this whole treatise.
When meaning is a pattern, the endeavor to construct an intelligent system becomes the design and analysis of the interplay between pattern detectors and generators.
Any intelligent system must have mechanisms to deal with and wield patterns of patterns that encode deeper, varied, and more ``meaningful'' representations.
Such mechanisms may be engineered rules, such as programmatic instructions, for example.
In this case, ``meaning'' is ascribed by the engineers who design the set of possible representations, decision paths, and outputs of the system.
And even though the system must output ``adaptive behaviors'', those behaviors, and the selection criteria to exhibit them, are rule-based in engineered systems.
Indeed, our adopted definition of intelligence affords that an intelligent system may be entirely designed.
We are interested, however, in learning systems that do not require imposed designs.

Since any intelligent system must have mechanisms to operate with patterns of patterns, when we compare designed/engineered systems to agents that learn, it follows that learning systems must \emph{create} inferences, context, and representations.
In short, any learning system will also be able to demonstrate its learning through the creation of patterns of patterns---both internal and external representations.
Therefore, in contrast to engineered systems where meaning is ascribed by the designers, \textbf{meaning is an emergent property of learning systems.}

Paradoxically, meaning-as-a-pattern is compatible with various quantitative modern information theories, including Shannon's, when meaning is a pattern that is encodable, decodable, and amenable to computation.
Therefore, while these information theories may not directly care about semantics, they can certainly utilize meaningful patterns \cite{adriaansCriticalAnalysisFloridi2010, adriaansInformation2020}.

We have been using the word \emph{pattern} quite liberally.
To be more specific, we will state that with respect to an intelligent system, \textbf{a pattern is a discernible entity}.
A pattern has a complement: every foreground has a background; the detection of signal implies a definition of what is not the signal; a plaid is described as repetitive, lined grids, which is different from other repetitive structures; etc.
From the intelligent system's point of view, ``the ability to perceive or infer information'' means something is only a pattern when it is perceptible by the system.
That is, it must differentiate the entity, becoming sensitive to it as some \emph{thing}.
The system must therefore contain (or develop) the sensors and their rules and relationships to intelligently operate with these ``things'', which may be even abstract, vague, internal concepts.

% Computationally, we will also subscribe to modern information theory notions that patterns are \emph{compressible} \cite{dennettRealPatterns1991, shannonMathematicalTheoryCommunication1948}.
% This means that the pattern can be represented in a more succinct representation.
% If the entity is already digitized, then there exists encodings that 

In a complex circuit composed of various sensors, each sensor may be sensitive to its own unique set of inputs.
In a signaling cascade, these inputs may include the transformed signals of preceding sensors in the circuit.
When the downstream sensor activates following some set(s) of transformed signals (and does not activate with other sets of signals), it has effectively ascribed meaning to those sets by recognizing the groupings as entities.
That is, the conditions to activate a sensor are semantic---they can be contrasted descriptively against the conditions that do not activate the sensor.

The above arguments seem reasonable in the ideal case of strict input signals followed by reliable sensor output.
It is not so clear how sensors and the larger system should accommodate noise.
Intelligence is about navigating a universe with incomplete information and noise.
We therefore recognize that noise and incomplete information are at least factors for external sensors and the system's memory capabilities.
We will tackle noise later in our development, but claim that the deeper sensors that process internal meta-information should also be resilient to noise and accommodate incomplete information because it permits the system to be more generally robust and adaptable.

In summary, \emph{information}, \emph{pattern}, and especially \emph{meaning} are very subjective terms relative to the intelligent system.
Nevertheless, we can describe any intelligent system and its processing of information as a cascade of patterns through sensors in this patterns-of-patterns framework.

\subsection{Intelligent system examples}

To help illustrate the \emph{patterns-of-patterns} framework in intelligent systems, it is worth considering some examples.

\subsubsection{Deep Neural Networks}

Deep neural networks (DNNs) illustrate our \emph{patterns-of-patterns} framework of an intelligent system in obvious ways.
The neurons of a DNN are sensors;\footnote{Note that many intelligent systems refer to their networked computational units as ``neurons''. While the role of neurons in these systems is to detect and direct signal, these systems employ neurons that perform drastically different computations. Importantly for our discussion, a neuron in a DNN is quite different from a spiking neuron like those used by the brain, despite the insistence from many in the Deep Learning community that that's ``how the brain works'' \cite{Believers2015, NeuralNetworksDeep}. This misconception persists due to the assumption that brain neurons signal downstream neurons through their rate, which, as we discuss later, is problematic.} the learned weights are its effective memory because a new input will trigger related, historical activations; and its rational outputs may include classifications, examples of possible futures (like sentence completions and deep fakes), or other transformations.

Each neuron in a DNN is sensitive to the activations of a subset of neurons in the previous layer.
The weights between the neurons in the previous layer and a target neuron in the next layer define how sensitive the target neuron is to each neuron in the previous layer.
The target neuron is effectively blind to the source neurons in the previous layer whose weights are near zero.
And for those neurons that can contribute because they have non-zero edges, the target neuron will respond differentially depending on how those source neurons activate.
The target neuron may only be sensitive to a particular subset of neurons being positive and another subset being negative, for example.
To that target neuron, it means something different when it is activated by the pattern vs. when it is not.
Via the backpropagation algorithm, the target neuron gets punished less as it gets better at honing relevant patterns and by how well the target neuron contributes to the different neurons in the next layer.
In DNNs, the ``ultimate'' meaning is ascribed by the optimization task designed by the engineer and implemented in the form of a loss function, but through learning, the neural network develops several internal representations of meaning.
Each neuron in the filtering cascade participates in the building of patterns of patterns.

As a more explicit example, let us consider the DNN, GoogLeNet \cite{szegedyGoingDeeperConvolutions2015}, used in image segmentation, as illustrated by \citet{olahFeatureVisualization2017}.
As that web page shows, different neurons are sensitive to different pixel patterns.
And furthermore, the deeper the layers, the more we see the groupings of earlier features: the first layers capture high-frequency edges, followed by textures, then grid-like repetitions, parts, and finally, objects.
The neurons of one layer aggregate various combinations of neurons in the previous layer.
Each neuron is sensitive to a particular, discernible set of neurons in the previous layer.

\subsubsection{TODO: Another example. Non-digital. \emph{Aplysia}?}

\subsection{Spikes, coupling, and neural coding strategies}

\emph{Spikes} are simply digitized events, a blip to signal something in time.
The term comes from electrophysiology recordings of nerve cells in animals.
When a source neuron fires an action potential to activate its target synapses, it induces an abrupt electrochemical change that is quite ``spiky'' compared to its non-firing state.
An example of spikes in electrophysiology recordings is given in Figure \ref{fig:kashiwadani_oscillations}, which shows the simultaneous recording from two electrodes in a rabbit's brain, each trace capturing the electrical activity of two different neurons.
In this example, many of the spikes of these two neurons oscillate and are synchronized as the animal inhales an odor.

\begin{SCfigure}[0.7][h]
\includegraphics[width=0.6\textwidth]{kashiwadani_oscillations}
\caption{Simultaneous extracellular electrophysiology recordings of two neurons from the rabbit olfactory bulb in response to odor stimulation. Arrows demarcate coincident spikes between these neurons. 
Image courtesy of \citet{kashiwadaniSynchronizedOscillatoryDischarges1999}.}
\label{fig:kashiwadani_oscillations}
\end{SCfigure}

%The encoding of patterns into binary streams for algorithmic use in digital circuits has led to humankind's greatest technological achievements.
%Digital representations permit general purpose computers to operate on bits divorced from their analog inputs.
%Through programming on a designed circuit, an imposed set of instructions ultimately applies Boolean algebra to perform mathematical logic on the bits to translate those digital representations into a computational result.
%As such, any ``meaning'' between the inputs and the outputs has been ascribed via the circuit's hardware engineers and software programmers.
%If we could follow the bits, we would see the application of various logic gates and mathematical transformations.
%These applied rules mean something to the designer; they show an intent of a particular outcome for a given input.
%Even in machine learning problems, the objectives and underlying computational architectures are human-designed.
%While Shannon ushered in the digitization of \emph{symbols} into binary streams, and these streams work across time---just ask anyone who has used a cell phone or ``streamed'' music or video---the computer systems to process these symbols is quite different from the architecture we propose, which digitizes \emph{events} with the specific aim to detect coincident events and derive meaning.
% It is an intelligent system, not a communication system.


In \emph{spiking neural networks} (SNNs) such as brains, the spiking activity of source neurons impacts the spiking activity of target neurons.
While each target neuron may be awash in a milieu of spiking activity from its source neurons, it is ultimately charged with emitting its own spikes in ways to modulate the system's behavior.
These blips that ``signal something using time'' are, therefore, each neuron's voice; the way it shows acknowledgement of signals received, integrating its understanding of the external environment as obtained primarily from spiking inputs, parsing meaningful spikes from irrelevant, and to nudge the system toward different output behaviors.
Spiking neurons are, therefore, perpetual decision-making relays that integrate and propagate signals through their chirps.

Spiking neurons may not be as passive as the description above alludes.
As dynamical machines, they may be arbitrarily complex and capable of integrating a variety of other external signals; not just spikes.
They may also have a variety of mechanisms to process incoming spikes differentially, being more sensitive to spikes from some sources as compared to others, or more sensitive with respect to the relative timing of spikes.
They may also have internal states that modulate their sensitivity to incoming spikes, perhaps becoming more sensitive to barrages of incoming spikes, for example.
Or even the opposite could be true, that they adapt and become numb to a barrage of inputs.
When they emit spikes, they might also be selective in their broadcasting, reliably signaling most spikes to some targets and only signaling to other targets under certain conditions.
And they may even have default active spiking behaviors, to maintain a default spiking frequency, for example, or to spike regularly like an oscillator.
In this regard, as compared to the activation functions of artificial, deep learning networks, which are typically utilized as passive filters, spiking neurons can be \emph{purposeful}.
Locally, each neuron might have particular optimal states of input and output behaviors and seek the environment that provides this optimal state.
And more globally, the spiking network/agent may seek environments that balances the ``desires''/optimal states of its population of neurons.
In short, spiking neurons use spikes as their most immediate mechanism to interpret and respond to their environment, but the way they integrate and incorporate these signals and emit spikes may be quite variable.

Even though neurons of animal brains use spikes as their most immediate communication protocol (other cell-signaling mechanisms can also be at play albeit at slower timescales), as previously noted, the brain probably uses spikes in a variety of ways.
The rationale for spikes being used in diverse ways is severalfold:
Principally, the brain has several types of neurons, each type with its own genetic makeup of ion channels and dendritic and axonal morphologies that result in different dynamical properties that provide a variety of input sensitivities and spiking characteristics.
Some cortical pyramidal neurons, for example, toggle between a bursting mode where they emit a series of rapid spikes under some conditions and a non-bursting mode where they fire one to a few spikes when activated \cite{zeldenrustNeuralCodingBursts2018}.
As well, particular types of neurons frequently form microcircuits that belong to distinct brain regions to carry out specific functions \cite{HandbookofBrainMicrocircuits}.
These microcircuits may generate their own rhythms and distinct spiking characteristics.
In short, brains are composed of many types of neurons with different types of connections, properties, and subnetworks that dictate their firing characteristics and capabilities.
As neuroscientists attempt the daunting task to decipher the neural code as emitted by neurons of the brain, they effectively take recordings of the brain under different conditions, accompanying stimuli, and known characteristics of the cells and the underlying network, then attempt to tease the information being transmitted \cite{bialekReadingNeuralCode1991, stanleyReadingWritingNeural2013}.

Measures of information often incorporate Shannon's notions of encoding and decoding. 
Spike encoding tasks are often expressed as a conditional probability: $P\left(y|x\right)$ where $y$ is the response---the spiking output of a neuron or some set of neurons, for example, and $x$ is the stimulus. 
In neuroscience studies, the stimulus is often external and may be far removed from the neurons being measured. 
For example, researchers may study the visual cortex of an animal in response to an image presented.
In this case, photons activate neurons of the retina that modulate perhaps thousands to millions of other neurons in a cascade that eventually modulates the cortical neurons being measured.
Decoding is cast as the inverse conditional probability: $P\left(x|y\right)$, the probability of the stimulus, $x$, given the observed response, $y$.
Together, Bayes formula, $P(x|y) = \frac{P(y|x)P(x)}{P(y)}$, can be applied in encoding/decoding paradigms.

While neuroscientists often cast experimental results into an encoding/decoding paradigm to help with their interpretations of recorded phenomena, it is difficult to find evidence that the brain really operates in this statistical realm \cite{bretteCodingRelevantMetaphor2019}.
In the Shannon sense, encoding is the transduction of a pattern and decoding is the process to recreate that input pattern at the other end.
But using statistics to find optimal compression and codec schemes, perhaps accommodating noise, is not the objective of organisms with brains or many intelligent systems.
In short, decoding is not usually the objective.
Instead, for animals, interfacing with the world in ways to help them gain and survive is the objective.
This means that the better these types of systems are at interpreting their inputs, possibly understanding themselves and their relationship to that world, and manipulating feasible responses, the more likely they will gain and survive.

\subsubsection{Coupling}

As noted above, the microcircuits and connectivity in a spiking neural network impacts its dynamics, so the network topology also plays a role in the interpretation of spiking neural codes.
There are many ways neurons of a spiking neural network can be coupled, but neural coupling broadly falls into two categories: \emph{directly} connected and \emph{functionally} connected.
In the context of the brain, we will say that a directly coupled network exists when neurons connect via synapses or, rarely, electrophysical coupling via gap junctions.
Functional connectivity exists between subsets of neurons when they participate in a patterned signaling cascade.
When the activity of an upstream neuron, not directly connected to a downstream neuron, influences the activity of that downstream neuron, we can say those neurons are functionally connected. 
Ultimately, all neurons must have specific subsets of direct connections to participate in a functional cascade.

% The "sequence of things" types of patterns are attractive and make sense. Any intelligent system is going to refer to history to imagine and predict a future. If we think of our own music listening experiences, it may take a few bars (a bar being itself a recognizable, temporal block of time that may also be further segmented into phrases and motifs) to acknowledge and get into the expected pattern, more affectionately known as the "groove". And during our listening experience, alterations to the groove are allowed and preferred if they do not deviate too strongly---perpetual repetitions will be deemed too monotonous and tuned out, and beats that are too varied will be considered cacophonous. Music plays that game of finding the "sweet spot" between monotony and cacophony to keep our attention. The better it does this, the more entertaining and musical we find it. Dance tunes may need the reliability of the beat, but may incorporate other changes to keep our attention.

% Indeed, we are very attuned to the "sequence of things". It's not just music. Something moving in the corner of our eye will divert our attention to it. We navigate the world with a sense of how it is shifting, how we expect it to behave, and with a sense of ourselves relative to it.

% The problem with the "sequence of things" patterns is that they take time. They require some interval of time to ponder and analyze the events that have transpired to dub them a particular kind of pattern. It is difficult to imagine how a single target neuron might differentiate even the simplest of temporal signals and how that could work in a broader network. The neuron could employ a "rate code" to differentiate fast vs. slow, but this requires an interval to evaluate. This delays the neuron's spiking output. If every neuron in a signaling cascade operated with such a rate code, the cascades could be terribly slow, prohibiting quick decisive actions. Indeed, the "sequence of things" patterns seem better suited for system-level recognition and responses, not to be performed at the individual neuron level. As described above, at the system level, the neurons that participate in such patterning will be *functionally* coupled. As also stated, every functional connection needs a substrate of direct connections, we need means of creating the direct connections at the individual level that enable the "sequence of things" patterns at the system level. Indeed, our goal is to derive a set of spike detection and emitting machines and the rules that permit such patterning to be an emergent property of self-organizing networks of spiking neurons.

 
 
\subsubsection{Rate coding}

While spikes may be encoded in several ways, neuroscientists frequently promote the \emph{rate code}.
This is because they observe neurons spiking faster following many stimuli.
In patch cell recording, for example, it is possible to inject electrical current into a single cell to induce it to fire rapidly.
But evidence of neurons firing faster or slower is also seen without direct electrode stimulation.
An early example came from \citet{hubelReceptiveFieldsSingle1959} where they recorded from single neurons in the cat V1 visual system showing that particular neurons spiked faster or slower depending on the angle and motion of a visible bar presented to the cat.
Certain neurons were sensitive to bars at a particular angle moving across the visual field and would spike faster when the bar was at the sensitive angle, and as the angle of the bar deviated from that angle, the neuron would spike less frequently.
This indicated a ``tuning curve'' for a neuron's response; that firing rates were indicative of the sensitivity to specific inputs to which the neuron was tuned.

Paradoxically, rate coding is problematic for systems operating in time.
While neurons of the brain have been observed to fire faster and slower under various conditions, within the context of a behaving system, the rate code is unlikely to be the primary readout for downstream neurons \cite{brettePhilosophySpikeRateBased2015, guoNeuralCodingSpiking2021}.
To illustrate the problem, imagine swatting at a fruit fly.
If your swipe is longer than 100 ms, the fly will likely escape.
Within that time---literally, the blink of an eye---the fly senses danger, makes an escape plan, jumps in a particular direction to get aloft, coordinates the flapping of its wings, and propels away \cite{fryeFlyFlightModel2001}.
Using some armchair calculations, even if all 100,000 neurons of the fly's brain were devoted to this evading behavior and each neuron was capable of firing at a very fast rate of 200 Hz, each neuron would emit between 0 and 20 spikes in the 100 ms window for its coding.
For simplicity, if we say that sensing danger, making an escape plan, jumping in a particular direction, and flapping wings are subsequent stages taking the same time, then that leaves 25 ms for each stage, resulting in 0-5 spikes from each neuron to encode the fly's actions. 
This also assumes that each stage can emit and induce the correct, coordinated action in one synaptic relay and does not need other intermediate layers of processing.
Indeed, it seems quite problematic for organisms that operate in real-time to adopt such rate codes because for a target neuron to evaluate fast vs. slow spiking from its inputs requires a window of time that delays its own output.
In a signaling cascade, these delays compound so complex behaviors such as those exhibited by the evading fruit fly would be impossible to carry out using purely rate coding.

\subsubsection{Other sequential patterning}

Earlier, we characterized \emph{pattern} as the ``particular arrangement or sequence of things''.
In the ``sequence of things'' camp, we can also imagine a single neuron emitting particular spiking patterns: fast, slow, rhythmic, or even something like Morse code.
But as just discussed, these patterns only induce delays for a downstream neuron listening to this neuron trying to discriminate its signals.

However, in this ``sequence of things'' camp, we can also imagine ensembles of neurons participating in a chain reaction. 
For example, given neurons labeled $A$, $B$, and $C$, if $A$ spikes, then $B$ spikes, and then $C$ spikes, then pattern $A-B-C$ could mean something different from the opposite activity, $C-B-A$. 
Or, perhaps $A$ chirps for a while before completing the pattern: $A-A-A-A-A-B-C$. So, the order of the sequence may matter, repetitions may matter, and durations could even matter.
Indeed, we recognize from our own experience and understanding of how brains direct actions that sequential spiking patterns must encode memories, future imaginations, and induce behaviors.
The evading fruit fly example illustrated above requires very complex, coordinated sequences of neuron signaling.
We argue that such sequential patterns are the result of \emph{functional} coupling that occurs through a system of \emph{directly} coupled connections.
That is, a target neuron is not monitoring several source neurons across relatively long time-windows to decide its output.
Rather, the target neuron is tuned for much more immediate responses to its directly connected sources; and the network is configured such that the neuron may participate in many arbitrarily complex temporal patterns at the system level operating through these functional cascades.

%In a small ensemble of neurons, we could imagine a neuron as a distinct drum used by a drummer who is knocking out specific rhythms.
%In a rock 'n roll beat, the sequence may include a ``snare'' neuron hitting the backbeat on 2 \& 4, and the ``cymbal'' neuron that spikes on the 1-beat every 4 bars.
%In this example, specific drum neurons spike at distinct, and largely repeatable and predictable intervals.
%We can imagine a ``rock 'n roll''-detecting neuron that listens to the drum set neurons for this pattern and spikes when it recognizes it.
%And there may be a ``blues''-sensitive neuron that listens to the same neurons but spikes when the ``high-hat'' neuron spikes on most beats and the ``cymbal'' is omitted.
%Either way, this encoding is plagued by delays.
%Furthermore, it is difficult to imagine how each beat-discriminating neuron would materialize, learn, and differentiate itself from the other beat-neurons.

\subsubsection{Coincidence as a ``particular arrangement'' pattern}

We develop a \emph{coincidence}-based code as an alternative to any code that involves long time-scale integrations such as the rate code.
In this alternative case, a target neuron is sensitive to particular populations of other neurons that are activated simultaneously.
As Figure \ref{fig:kashiwadani_oscillations} shows, in animal brains, some neurons that respond to the same stimulus will fire rapidly (in this example, around 30 Hz), but moreover, it is the synchronous behavior that may propagate signal to downstream targets.
As a simplistic illustration, a target neuron that has synaptic connections from these two input neurons may itself only be activated when these two neurons spike together; shift one of these spike trains out of phase to the other so that they do not synchronize and the target neuron would remain silent.
In this paradigm, the target neuron could immediately fire an action potential following the first set of coincident spikes that it detects.
Coincidence---at least for positive, activating spikes---is therefore an AND operation.
The role of a series of subsequent coincident spikes as observed in Figure \ref{fig:kashiwadani_oscillations} may be to continue to activate the downstream neuron and encode the duration of the AND signal.
The repeated synchronized spikes may also add insurance that the signal is propagated.

% Coincidence and any pattern is amenable to compression.

Going back to our definition of information---what is conveyed or represented by a pattern---we have discussed ways of interpreting and representing patterns with spikes.
When two or more neurons share a common, driving signal, then their output spikes will tend to co-occur more often than by chance. 
That must mean something. 
Something is triggering these neurons to behave similarly. 
To derive that meaning and modify system behavior if need be, we need a target neuron to connect to those neurons that sporadically fire together, to recognize when they are synchronized, and respond accordingly.
The target neuron, in turn, can participate in subsequent downstream signaling with other neurons that are sensitive to other  signals that happen to be co-occurring.
It is in these ways that associations can be made to integrate many elements in a dynamic context.

\subsection{Sensors and the digitization of events}

But why spikes?
How and why did biology come upon spikes and evolve nervous systems and all they comprise (hemispheres, spinal cords, nerve cells and their neurite growth, dendritic trees and spines, axonal projections and myelin sheaths, electrochemical properties, synaptic formation, the myriad of neuron types distributed in particular regions within a dynamic network to form microcircuits, etc., as well as supporting cells like glia and anatomical features like cerebral-spinal fluid, the blood-brain barrier, and skulls to protect it)?
The human brain consumes 2/3rds of the energy used by the body \cite{TODO} (which is still only as much energy as a 40-watt lightbulb \cite{TODO}).
All of this effort and complexity is directed at spike communication and processing.
But are spikes simply an evolutionary kludge?
That is, as with many components of biological organisms, simple mechanisms are coopted and jury-rigged into ever more complex systems that are more evolutionarily advantageous than their ancestors' solution, but nevertheless, may fall short of an optimal design \cite{dawkinsBlindWatchmakerWhy2015}.
Do spikes fall into that category?å
Does the complexity of the nervous system indicate the importance and diverse possibilities of spikes, or is this complexity the outcome of a slow accrual of haphazard computational mechanisms available given biophysical, chemical, cellular, and evolutionary constraints?

Any answers to the questions above will not only help us better understand the brain, but they will also assist with bio-inspired designs. 
Nature evolved various means for locomotion---numerous types of wings, legs, and slithering ways to get around---but humans still invented the wheel.
Analogously, nature evolved the spike and humans invented bits.
We do not yet understand the computational capabilities of spikes to determine where they might be used over computational architectures using bits.
Broadly speaking, spikes are rather imprecise.
Sure, the blip/moment of the spike may be precise, but like bits, what they represent is abstract.
And in the case of spikes, the timing could be important.
A spike shifting 1 millisecond (or attosecond!) in time may be critical to the computations that use it, or it may not.
While computing with spikes or bits are inevitably areas of neverending research, when it comes to spikes, we currently lack a basis to build a computational framework.
In this section, we aim to establish such a basis.

%What may be some of the basic principles that natural selection utilized to evolve sensory systems to digitize events and to build this organ called the brain to communicate and operate via spikes?
%Sure, at a high level, we assume that an intelligent system, especially one operating in real-time, will seek to accrue and respond to information in a quick, nuanced way.
%So it is not so surprising that multicellular organisms evolved external sensors across their surfaces and developed electrical networks to conduct the processing.
%But there is nothing in this formulation to indicate that spikes are anything more than an evolutionary kludge---that, as with many components of biological organisms, simple mechanisms are coopted and jury-rigged into ever more complex systems that are more evolutionarily advantageous than their ancestors' solution, but nevertheless, fall short of an optimal design \cite{dawkinsBlindWatchmakerWhy2015}.
%To be more explicit, our question is, have nervous systems and all they comprise---hemispheres, spinal cords, their neurite growth, dendritic trees and spines, axonal projections, electrochemical properties, synaptic formation and learning, and myriad of neuron types distributed in particular regions within a dynamic network---all directed at spike communication and processing, was all of this complexity due to a slow, haphazard accrual of computational mechanisms available given biophysical, chemical, cellular, and evolutionary constraints; or are there properties that are generically special about digitized events as compared to other digitized or continuous functions for intelligent systems?

%From an engineering perspective, we would like to understand spike processing to differentiate it from circuits that process bits via transistors.
%Spikes are to bits as legs are to wheels.
%Nature evolved spikes and legs, but not bits and wheels.
%Both legs and wheels may be used for locomotion, but their utility and particular application will be very much dependent on context.
%Similarly, we want to determine the contexts and ways to optimally compute with spikes vs. bits.

\subsubsection{Events}

% \cite{humphriesSpikeEpicJourney2021}%
%
%Indeed, there are various studies that propose physical mechanisms underlying the evolution of the action potential \cite{sakaryaPostSynapticScaffoldOrigin2007, pinedaEvolutionActionPotential2010, liebeskindEvolutionSodiumChannels2011, brenesSystemMethodIncreasing2016}.
%As well, the development of the nervous system, neurite growth, dendritic spines, and synaptic formation and learning, 
%but these do not speak generally to the selection pressures to evolve a nervous system.

As stated earlier, spikes are blips to signal something in time.
It might be more accurate to say ``something \emph{using} time.''
Both expressions apply, but \emph{in} time connotes something more timely and relevant in the moment.
This may often be the case in biological brain functions and real-time systems, but \emph{using} time permits time to be an arbitrary axis to portray positive bits.
That is, historical time is a substrate of the encoding, but one could replace time with another dimension, like 1D space, and our general theory would hold.
While \emph{using} time may be generically applicable to our theory, we will remain with \emph{in} time for its more obvious and relevant notions of systems operating in real-time.
Therefore, spikes are encodings of \emph{events}, where an event is an encapsulation of a pattern along one dimension, time.

\subsubsection{Sensors that recognize events and emit spikes}

In a spiking neural network, the neurons are sensors that recognize particular temporal events and emit a spike to connote their recognition.
But what does it mean to recognize an event?
And why should spikes be the output?
Let us first consider the recognition of an event.
We can cast any cause-and-effect paradigm as sequential stages of two temporal patterns.
The effect operates as a \emph{sensor} to the cause---without a cause, the effect does not happen; and when the effect is observed, it indicates the presence of a cause.
(Granted, there may be several different causes that induce a particular effect and a single cause that can produce several different effects.
We will ignore the many-to-many issue for the moment.
As well, we will currently ignore probabilistic effects where the fidelity of the effect following a response is not perfectly realized---where the effect may occur spontaneously without the cause, be delayed, have different characteristics with each repeated trial of the cause, etc.
For the purposes of the current discussion, some cause is adequately defined and induces some effect, which is also adequately defined; it is deterministic.)
We will adopt propositional logic 

With $P_{i} \rightarrow Q_{j}; i,j \in \mathbb{R}$

\subsection{Causes and Effects}

We can cast any cause-and-effect paradigm as sequential stages of two temporal patterns.
Our ability to ascribe a label to each event, one for ``cause'' and one for ``effect'', acknowledges each event as a distinct pattern/entity.
Additionally, given our semantic understanding of these labels, we also understand the relational pattern that \emph{cause} precedes and induces \emph{effect}.
Because of this relationship, we actually have 3 patterns: 1) the first event, which is the cause; 2) the second event, the effect; and 3) the rule that this particular cause pattern induces the effect pattern.
(Anything that can be described, including rules, exists as an entity.)
It is the third pattern that is necessary for intelligence.
It is the ``pattern of patterns''; the pattern that ascribes meaning between the two events.

The sensing and encoding of these patterns occurs in the intelligent agent.


In this paradigm, to a remote observer, 
Again, an intelligent system must acquire patterns of patterns.


Granted, it could be that one event follows another completely independently and because they occurred independently, theyand because thsoe two events are independent, they are 


In a cause-and-effect paradigm, the cause may be opaque to a remote observer.
In this case, to an outside observer, only the effect is capable of being monitored while the cause is indiscernible.
However, from the perspective of the effect, the cause is discernible---the effect happened because it responded to the cause.
The rules governing the effect's response can be viewed as its sensing capabilities.
``I tripped and fell down'' is a cause and effect.
My tripping caused the effect of me falling down.
If you, an external observer, see me falling down, you may have not seen me stub my toe on the sidewalk crack (the cause).
The falling down pattern is the behavior my body goes into when it obeys the rules of gravity. 

when the cause is recognized as present, the effect will happen and when it is not recognized as present, then the effect will 
This effect-as-causal-sensor argument 
\subsubsection{Digitization of events}


%As well, each cause and effect may only be slightly discernible to our remote observer.
%In these cases where they are ill-defined.
%The vocabulary and grammars available to operationalize anything given these patterns may be lacking.



They are discernible in the sense that we can discuss them as separate entities, but, admittedly, both may be ill-defined without the vocabulary and grammars to really operationalize anything given these patterns.


Furthermore, we can say that the effect serves as a sensor to the cause; for the effect to be a thing indicates a probable cause.
There may be several causes that can induce the same effect.
And there may even be a set of unknown causes that induce an effect.
Perhaps the causes are completely unknown.



The effect, which is some sort of change, indicates the presence of a probable cause, 
while no effect indicates the absence of such a stimulus.
In short, we can label any stimulus-response, probabilistic toggle, or whatever expression we want to use to describe a cause-and-effect event as a pattern.


\subsection{Overview of our approach}

We have presented several concepts above that demand further development.
While we have incorporated understandings of the brain as obtained through neuroscience, our objective is not to reverse-engineer the brain.
Instead, we digitize events into spikes and ask how we can build the computational units to recognize and emit spikes, configured in a network that behaves intelligently.
In this dynamical system, we desire some units to emit their spikes together when they need to, but then to de-synchronize when necessary, followed by some of those neurons to synchronize with other subsets in subsequent signaling.
